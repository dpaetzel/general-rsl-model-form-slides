\documentclass[uni]{ocbeamer}


\title{General RSL model form}


\date{2 November 2022}
\author{\alert{David Pätzel}}
\institute{University of Augsburg, Germany}


\usepackage[british]{babel}


\usepackage{csquotes}


\usepackage[beamer,customcolors,markings]{hf-tikz}


\usepackage{tikz}
\usetikzlibrary{arrows}
\usetikzlibrary{calc}
% For brace, see https://latexdraw.com/how-to-draw-curly-braces-in-tikz/ .
\usetikzlibrary{decorations.pathreplacing,calligraphy}


\usepackage[copyright]{ccicons}


\usepackage{amsfonts}


\newcommand{\normal}{\mathcal{N}}
\newcommand{\const}{\text{const.}}
\newcommand{\RR}{\mathbb{R}}
\newcommand{\BB}{\mathbb{B}}
\newcommand{\TT}{\mathbb{T}}
\newcommand{\FF}{\mathbb{F}}
% data
\newcommand{\DD}{\mathcal{D}}
\newcommand{\XX}{\mathcal{X}}
\newcommand{\YY}{\mathcal{Y}}
% models
\newcommand{\MM}{\mathcal{M}}
\newcommand{\ff}{\hat{f}}
% populations
\newcommand{\PP}{\mathbb{P}}
% variational inference
\newcommand{\LL}{\mathcal{L}}
\newcommand{\EE}{\mathbb{E}}
\newcommand{\lnpmd}{\ln p(\MM \mid \DD)}

\newcommand{\MAE}{\text{MAE}}
\newcommand{\XCSF}{\text{XCSF}}
\newcommand{\BERBL}{\text{BERBL}}
\newcommand{\lit}{\text{lit.}}
\newcommand{\modular}{\text{mod.}}


\usepackage{marvosym}


\newcommand{\shell}{{\footnotesize\color{gray}\syfamily 🐚}}
\newcommand{\Shell}{{\Huge\syfamily 🐚}}
\newcommand{\browser}{{\footnotesize\color{gray}\syfamily 🌎}}
\newcommand{\Browser}{{\Huge\syfamily 🌎}}
\newcommand{\editor}{{\footnotesize\color{gray}\syfamily 📄}}
\newcommand{\Editor}{{\Huge\syfamily 📄}}


\begin{document}


\begin{frame}{General RSL model form}
  Popular RSLs (e.\,g.\@ XCS, XCSF, UCS, BioHEL, …) build \alert{discriminative function models} of the form

  \vspace{0.3cm}

  % tikzmarkin{a}(below right offset)(above left offset)
  \begin{equation}
    \ff_{
      \hfsetfillcolor{green!10}
      \hfsetbordercolor{green}
      \tikzmarkin<1->{MM}(0.03,-0.1)(0,0.3)
      \MM
      \tikzmarkend{MM}
    }(\theta, x) =
    \sum_{k = 1}^{K}
    \quad
    \hfsetfillcolor{green!10}
    \hfsetbordercolor{green}
    \tikzmarkin<1->{mk}(0.05,-0.2)(-0.1,0.5)
    m_{k}(x)
    \tikzmarkend{mk}
    \quad
    \hfsetfillcolor{blue!10}
    \hfsetbordercolor{blue}
    \tikzmarkin<1->{gamma}(0.05,-0.2)(-0.1,0.5)
    \gamma_{k}
    \tikzmarkend{gamma}
    \quad
    \hfsetfillcolor{blue!10}
    \hfsetbordercolor{blue}
    \tikzmarkin<1->{ff}(0.05,-0.2)(-0.1,0.5)
    \ff_{k}(\theta_{k}, x)
    \tikzmarkend{ff}
  \end{equation}

  \only<1->{
    \begin{tikzpicture}[remember picture,overlay]
      \node (mstruc) at ($(MM) + (-0.7, -1)$) {\color{green}model structure};
      \node (resp) at ($(mk) + (-0.25, -1.5)$) {\color{green}responsibility};
      \node (mix) at ($(gamma) + (0.1, -2)$) {\color{blue}mixing weight};
      \node (subm) at ($(ff) + (1.3, -1.5)$) {\color{blue}submodel};

      \coordinate (mstruc-MM) at ($(MM) + (0.2, -0.4)$);
      \draw[-stealth,thick,green] (mstruc) -- (mstruc-MM);

      \coordinate (resp-mk) at ($(mk) + (0.5, -0.7)$);
      \draw[-stealth,thick,green] (resp) -- (resp-mk);

      \coordinate (mix-gamma) at ($(gamma) + (0.25, -0.7)$);
      \draw[-stealth,thick,blue] (mix) -- (mix-gamma);

      \coordinate (subm-ff) at ($(ff) + (0.75, -0.7)$);
      \draw[-stealth,thick,blue] (subm) -- (subm-ff);

      % \draw<3->[pen colour={orange},very thick,decorate,decoration={calligraphic brace,amplitude=5pt}] ($(mk) + (-0.05, 0.3)$) --  ($(ff) + (1.5, 0.3)$) node [above=3pt,pos=0.5] {\color{orange}rule $k$};
    \end{tikzpicture}
  }

  % Ugly hack to work around uncover/visible/invisible not working with hf-tikz (we have to use only for the tikzpickture and thus everything shifts around).
  \only<1>{
    \vspace{1.7cm}
  }

  \only<1->{
    \vspace{1.3cm}
  }

  {
    \color{gray}
    where $\theta = \left\{ \left\{ \theta_{k} \right\}_{k=1}^{K}, \left\{ \gamma_{k} \right\}_{k=1}^{K} \right\}$
    and $\MM = \left\{K, \left\{ m_{k} \right\}_{k=1}^{K} \right\}$.
  }

  \begin{tikzpicture}[remember picture,overlay]
    \node [anchor=south] at (current page.south) {
      \tiny
      \color{gray}
      \copyright{} 2022 David Pätzel
      \ccbync{}
    };
  \end{tikzpicture}
\end{frame}


\end{document}
